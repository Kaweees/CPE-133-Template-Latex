\documentclass[12pt, letterpaper]{article}
\usepackage[utf8]{inputenc}
\usepackage{sectsty}
\usepackage[compact]{titlesec}
\usepackage[left=0in,right=0in,margin=1in]{geometry}
\usepackage{booktabs}
\usepackage{tabu}
\usepackage{siunitx}
\usepackage[labelfont=bf, skip=5pt, font=small]{caption}
\usepackage{subcaption}
\usepackage{graphicx}
\usepackage{fancyhdr}
\usepackage[style=chem-acs, articletitle=true]{biblatex}
\addbibresource{references.bib}
\usepackage{float}

\setlength{\parskip}{1em}
\titlespacing{\section}{0pt}{5pt}{0pt}
\sectionfont{\fontsize{15}{15}\selectfont}

\begin{document}
\begin{center}
  \large{\textbf{Lab Report Name}} \\[10pt]
  \large{Quarter Year, CPE 133-01} \\[5pt]
  \large{Miguel Villa Floran} \\[5pt]
  \large{Partners: First Last, First Last}
\end{center}

\noindent\rule{\textwidth}{0.5pt}

\section*{\makebox[6.45in]{Summary}}

Description of what you built and what it does. A schematic, like the one in Figure 1, would be a good thing to have in here. As would some Boolean equations that describe the circuits function, like we see in \ref{eq:1}.

\begin{equation}
  Y = \overline{A}B + AB
\end{equation}

\noindent\rule{\textwidth}{0.5pt}

\section*{\makebox[6.45in]{Verification}}

\noindent Use this section to tell me how you tested your circuit. Did you simulate it? What input combinations did 
you use? How did you test on the FPGA? Why you think the test cases adequately test the circuit.
about what it is you built.

\noindent\rule{\textwidth}{0.5pt}

\section*{\makebox[6.45in]{Answers to Questions}}

\noindent Some lab write-ups will include questions.  Please include the questions themselves and your answers to 
these questions here.   

\noindent\rule{\textwidth}{0.5pt}

\section*{\makebox[6.45in]{Code}}

\noindent You can upload your code with your report on PolyLearn.

\noindent\rule{\textwidth}{0.5pt}

\printbibliography[title={\centering References}]

\end{document}